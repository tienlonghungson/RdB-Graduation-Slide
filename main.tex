\documentclass{beamer}
\usepackage{header/packages}

%gets rid of bottom navigation symbols
%\setbeamertemplate{navigation symbols}{}

%gets rid of footer
%will override 'frame number' instruction above
%comment out to revert to previous/default definitions
%\setbeamertemplate{footline}{}

% Tác giả, Tiêu đề, vân vân
\title[]{{\huge \bf Graduation Thesis} \\
\large Run length limited de Bruijn sequences for quantum communication}
\author[Nguyen Tien Long]{
Nguyen Tien Long, CTTN-CNTT-K63% \inst{1} 
}

\institute[]{
%\inst{1}% 
Computer Science Department\\
School of Information and Communication Technology.
}



% \logo{\includegraphics[scale=0.05]{BVP-logo bk-rgb.jpg} \vspace{220pt}}
\pgfplotsset{compat=1.18}
\begin{document}

\begin{frame}
    \titlepage
\end{frame}

\begin{frame}
    \frametitle{Table of Contents}
    \tableofcontents
\end{frame}

\section{Introduction}
\begin{frame}{Coding Theory}
    \begin{itemize}
        \item The study of the properties of codes and their respective fitness for specific applications.
    \end{itemize}
    \begin{columns}
        \begin{column}{0.5\textwidth}
            \begin{figure}
            \centering
                \includegraphics[width=0.8\textwidth]{Images/Introduction/CodingTheory.png}
                \caption{Coding Process.}
                \label{fig:coding_process}
            \end{figure}
        \end{column}
        
        \begin{column}{0.5\textwidth}
            \begin{figure}
                \centering
                \includegraphics[width=\textwidth]{Images/Introduction/sourceandchannelcoding.png}
                \caption{Source and Channel Coding.}
                \label{fig:sourcechannelcoding}
            \end{figure}    
        \end{column}    
    \end{columns}
\end{frame}

\begin{frame}{Coding Theory - Applications}
    \begin{columns}
        \begin{column}{0.5\textwidth}
            \begin{itemize}
                \vfill \item Run length limited code is used in CD, DVD.
                \begin{figure}
                    \centering
                    \includegraphics[width=0.7\textwidth]{Images/Introduction/CDDVD.png}
                    % \caption{Caption}
                    % \label{fig:my_label}
                \end{figure}
                \hfill
                \vfill \item 3G,4G networks use Reed- Solomon code.
                \begin{figure}
                    \centering
                    \includegraphics[width=0.7\textwidth]{Images/Introduction/3G4G.png}
                    % \caption{Caption}
                    % \label{fig:my_label}
                \end{figure}
                \vfill \item 5G networks use Turbo code.
                \begin{figure}
                    \centering
                    \includegraphics[width=0.7\textwidth]{Images/Introduction/5G.png}
                    % \caption{Caption}
                    % \label{fig:my_label}
                \end{figure}
            \end{itemize}
        \end{column}
        
        \begin{column}{0.5\textwidth}
            \begin{itemize}
                \vfill \item LDPC code is used in Data modems, telephone transmissions, and the NASA Deep Space Network.
                \begin{figure}
                    \centering
                    \includegraphics[width=0.8\textwidth,height=0.6\textheight]{Images/Introduction/NASA.png}
                    % \caption{Caption}
                    % \label{fig:my_label}
                \end{figure}
            \end{itemize}
        \end{column}
    \end{columns}
\end{frame}


\begin{frame}{Constrained Code}
    \begin{itemize}
        \vfill \item Code: set of (binary) sequences, strings.
        
        $\rightarrow$ Constrained code $\mathbb{C}$: set of sequences satisfying given constraints.
        \vfill \item Presentation: a directed graph $G=(V,E)$.
        \vfill \item Path: sequence of edges $\left(e_{1},e_{2},\ldots,e_{n}\right)$ such that the end vertex of $e_{i}$ is the start vertex of $e_{i+1}$ $e_{i}\in E$. 
        \vfill \item Simple path: path with no repeated edges.
    \end{itemize}
\end{frame}

\begin{frame}{De Bruijn Code (Positioning Code)}
    \begin{columns}
        \begin{column}{0.65\textwidth}
            \begin{enumerate}
                \item De Bruijn graph of order $k$, $G_{k}$:
                \begin{itemize}
                    \item Each vertex is labeled by a sequence of length $k-1$.
                    \item A directed edge from $\bfx = \left[x_{0}x_{1}\ldots x_{k-2}\right]$ to $\bfy=\left[y_{0}y_{1}\ldots y_{k-2}\right]$ $\Leftrightarrow\ x_{1}x_{2}\ldots x_{k-2}=y_{0}y_{1}\ldots y_{k-3}$. 
                \end{itemize}
                \item De Bruijn sequence:
                \begin{itemize}
                    \item A cyclic binary de Bruijn of order $k$ is a length $2^{k}$ sequence such that each length $k$ string appears exactly once.
                    \item Longest simple path in de Bruijn graph (Eulerian cycle) $\equiv$ de Bruijn sequence.
                \end{itemize}
                
                \item Applications:
                \begin{itemize}
                    \item Cryptography.
                    \item Interconnection networks.
                    \item VLSI testing.
                    \item Combine with biology: DNA storage.
                \end{itemize}
            \end{enumerate}
        \end{column}
        
        \begin{column}{0.35\textwidth}
            \begin{figure}
                \centering
                \includegraphics[width=0.8\textwidth]{Images/deBruijnOrder4.png}
                \caption{de Bruijn graph of order $4$.}
                \label{fig:dB4}
            \end{figure}
        \end{column}
    \end{columns}
\end{frame}

\begin{frame}{What do we concern about a code?}
    \begin{block}<1>{Rate}
        Rate and Maximal asymptotic rate tend to as high as possible.
    \end{block}
    \begin{block}<1>{Efficient encoding algorithm}
        Encoding algorithm: maps messages into codewords, generates sequences in the code, \dots.
    \end{block}
    \begin{block}<1>{Efficient decoding algorithm}
        Decoding algorithm: maps codewords into messages, locates the position of the sequences in the ordered code, \dots.
    \end{block}
    \begin{block}<0>{Et Cetera}
        Fault tolerant, robust positioning, \dots.
    \end{block}
\end{frame}

\section{Motivation}
% \begin{frame}{Satellite Quantum Key Distribution (QKD)}
%     \visible<1->{
%         Commercial QKD: deployed over optical fibre. But $\mathrm{range}< 1000 \mathrm{km}$
%     }
    
%     \visible<2->{
%         $\Rightarrow$ Satellite QKD: alternative method to establishing intercontinental secure communication links. Challenges:
%         \begin{itemize}
%             \item High channel losses.
%             \item Rapid relative motion between the transmitter and receiver.
%         \end{itemize}
%     }
    
%     \visible<3->{
%         $\Rightarrow$ A classical channel is used along to synchronise quantum channel~\sfcite{duan2021survey,khader2018time}.
%         \begin{figure}
%             \centering
%             \includegraphics[scale=.3]{Images/Motivation/SatteliteQKD.png}
%             \caption{High-level satellite Quantum Key Distribution schematic.}
%             \label{fig:satelliteQKD}
%         \end{figure}
%     }
% \end{frame}

\begin{frame}{Satellite Quantum Key Distribution (QKD)}
    Commercial QKD: deployed over optical fibre.
    \begin{overprint}
        \onslide<1>
        But: $$\mathrm{range}< 1000 \mathrm{km}$$
        \onslide<2>
        But: $$\mathrm{range}< 1000 \mathrm{km}$$
        $\rightarrow$ Satellite QKD: alternative method to establishing intercontinental secure communication links.
        
        \onslide<3> 
        But: $$\mathrm{range}< 1000 \mathrm{km}$$
        $\rightarrow$ Satellite QKD: alternative method to establishing intercontinental secure communication links.
        
        Challenges:
        \begin{itemize}
            \item High channel losses.
            \item Rapid relative motion between the transmitter and receiver.
        \end{itemize}
        \onslide<4-> 
        % \visible<4->{
        $\rightarrow$ Satellite QKD: alternative method to establishing intercontinental secure communication links.
        
        $\rightarrow$ A classical channel is used along to synchronise quantum channel~\sfcite{duan2021survey,khader2018time}.
        \begin{figure}
            \centering
            \includegraphics[scale=.3]{Images/Motivation/SatteliteQKD.png}
            \caption{High-level satellite Quantum Key Distribution schematic.}
            \label{fig:satelliteQKD}
        \end{figure}
        % }
    \end{overprint}
    
    % \visible<3->{
    %     $\Rightarrow$ A classical channel is used along to synchronise quantum channel~\sfcite{duan2021survey,khader2018time}.
    %     \begin{figure}
    %         \centering
    %         \includegraphics[scale=.3]{Images/Motivation/SatteliteQKD.png}
    %         \caption{High-level satellite Quantum Key Distribution schematic.}
    %         \label{fig:satelliteQKD}
    %     \end{figure}
    % }
\end{frame}


\begin{frame}{Related Work}
    \citeauthor{zhang2021timing}\sfcite{zhang2021timing}: de Bruijn based timing-synchronization system (dBTS).
    \begin{overprint}
        \onslide<2-4> 
            \begin{columns}
            \visible<2->{
                \begin{column}{.37\textwidth}
                    \begin{centering}
                        \fcolorbox{red}{white}{Encode}
                        \begin{figure}
                            % \centering
                            \includegraphics[width=0.6\textwidth]{Images/Motivation/SatelliteEncode.png}
                            % \caption{Caption}
                            % \label{fig:my_label}
                        \end{figure}
                    \end{centering}
                    \begin{itemize}
                        \item Linear feedback shift register (LFSR): generate an order $k$ de Bruijn sequence.
                    \end{itemize}
                \end{column}
                \vrule{}
            }
            \visible<3->{
                \begin{column}{.26\textwidth}
                    \begin{centering}
                        \fcolorbox{red}{white}{Noisy channel}
                        \begin{figure}
                            % \centering
                            \includegraphics[width=0.6\textwidth]{Images/Motivation/Cloud.png}
                            % \caption{Caption}
                            % \label{fig:my_label}
                        \end{figure}
                    \end{centering}
                \end{column}
                \vrule{}
            }
            \visible<4->{
                \begin{column}{.37\textwidth}
                    \begin{centering}
                        \fcolorbox{red}{white}{Decode}
                        \begin{figure}
                            % \centering
                            \includegraphics[width=0.5\textwidth]{Images/Motivation/GSDecode.png}
                            % \caption{Caption}
                            % \label{fig:my_label}
                        \end{figure}
                    \end{centering}
                    \begin{itemize}
                        \item Look-up table: locate the position of a length $k$ subsequence in the whole sequence.
                    \end{itemize}
                \end{column}
            }
            \end{columns}
        \onslide<5->
            \visible<5->{
                \begin{block}{Transmit a modulated sequence}
                    \begin{itemize}
                        \item {\color{red}Constraint}: avoid long period of no pulse .
                        \item {\color{red}Requirement}: positioning sequence.
                        \item {\color{red}Method}: de Bruijn sequence, pulse modulation: $1\rightarrow\mathrm{on-on},\ 0\rightarrow\mathrm{on-off}$, called Hybrid de Bruijn (HdB) sequence.
                    \end{itemize}
                \end{block}
            }
            \visible<6->{
                \begin{alertblock}{Drawback}
                    \begin{itemize}
                        \item $\mathrm{Rate}=0.5$.
                        \item Encode: LSFR (prerequisite: suitable primitive polynomial).
                        \item Decode: use look-up table (exponential complexity).
                    \end{itemize}
                \end{alertblock}
            }
    \end{overprint}
    
\end{frame}

\begin{frame}{Contributions}
    \begin{block}{Propose a new combinatorial object RdB}
        \begin{itemize}
            \item Can be encoded and decoded efficiently.
            \item Can replace HdB sequence: higher rate and maximal asymptotic rate, more general and adaptive.
            \item Potential application: DNA storage.
        \end{itemize}
    \end{block}
    \begin{block}{Determine the maixmal length of RdB}
        \begin{itemize}
            \item Explicit formula.
        \end{itemize}
    \end{block}
    \begin{block}{Encoding Algorithm}
        \begin{itemize}
            \item Constant amortized time per symbol.
        \end{itemize}
    \end{block}
    \begin{block}{Decoding Algorithm}
        \begin{itemize}
            \item Sub-linear time with respect to the length of the sequence.
        \end{itemize}
    \end{block}
\end{frame} 

\section{Run length limited de Bruijn sequence - Graph Presentation}
\begin{frame}{De Bruijn Sequence (Positioning Sequence)}
    \begin{columns}
        \begin{column}{0.65\textwidth}
            \begin{enumerate}
                \vfill\item De Bruijn sequence:
                \begin{itemize}
                    \vfill\item A cyclic binary de Bruijn of order $k$ is a length $2^{k}$ sequence such that each length $k$ string appears exactly once.
                    \vfill\item Example: A de Bruijn sequence of order $4$.
                    
                    {\color{teal}Cyclic} : {\color{blue}0000100110101111}.

                    {\color{teal}Acyclic} : {\color{blue}0000100110101111000}.
                
                    \vfill\item De Bruijn sequence $\equiv$ longest simple path in de Bruijn graph (Eulerian cycle).
                    
                    
                \end{itemize}
                \visible<2->{
                \vfill\item De Bruijn graph of order $k$, $G_{k}$:
                \begin{itemize}
                    \vfill\item Each vertex is labeled by a sequence of length $k-1$.
                    \vfill\item A directed edge from $\bfx = \left[x_{0}x_{1}\ldots x_{k-2}\right]$ to $\bfy=\left[y_{0}y_{1}\ldots y_{k-2}\right]$ $\Leftrightarrow\ x_{1}x_{2}\ldots x_{k-2}=y_{0}y_{1}\ldots y_{k-3}$. 
                \end{itemize}
                }
            \end{enumerate}
        \end{column}
        
        \visible<3->{
        \begin{column}{0.35\textwidth}
            \begin{figure}
                \centering
                \includegraphics[width=0.75\textwidth]{Images/deBruijnOrder4.png}
                \caption{de Bruijn graph of order $4$.}
                \label{fig:dB4}
            \end{figure}
        \end{column}
        }
    \end{columns}
\end{frame}

\begin{frame}{Run Length Limited de Bruijn (RdB) sequences}
    \begin{definition}
        A $(k,s)$-RdB sequence: a de Bruijn sequence of order $k$ containing at most $s$ consecutive bit $0$'s.
    \end{definition}
    \begin{overprint}
        \onslide<1> 
            Trivial cases:
            \begin{itemize}
                \item $s\geq k$: original de Bruijn sequence.
                \item $s=k-1$: remove $0^{k-1}$ in the de Bruijn graph.
            \end{itemize}
            $\Rightarrow$ Interested in:
            \begin{itemize}
                \item $s<k-1$: eliminate vertices with more than $s$ consecutive bit $0$'s.
            \end{itemize}
        \onslide<2->
            \begin{columns}
            \visible<2->{
                \begin{column}{0.33\textwidth}
                    \begin{figure}[htbp]
                        \centering
                        \includegraphics[scale=0.27]{Images/deBruijnOrder4.png}
                        \caption{de Bruijn graph of order $4$.}
                        % \label{fig:my_label}
                    \end{figure}
                \end{column}
            }
            \visible<3->{
                \begin{column}{0.34\textwidth}
                    \begin{figure}[htbp]
                        \centering
                        \includegraphics[scale=0.35]{Images/RdB/RdB_4_3.png}
                        \caption{$(4,3)$-RdB graph.}
                        \label{fig:RdB_4_3}
                    \end{figure}
                \end{column}
            }
            \visible<4->{
                \begin{column}{0.33\textwidth}
                    \begin{figure}[htbp]
                        \centering
                        % \includegraphics[scale=0.4]{Images/RdB/RdB_4_1.png}
                        \includegraphics[scale=0.26]{Images/RdB/4_1_RdB.png}
                        \caption{$(4,1)$-RdB graph.}
                        \label{fig:RdB_4_1}
                    \end{figure}
                \end{column}
            }
        \end{columns}
    \end{overprint}
    
    
\end{frame}

\begin{frame}{Maximal length of $(k,s)$-RdB sequence}
    Given $k,s$. Notations:
    \begin{itemize}
        \item $\ell(k,s)$: maximal length of simple path $(k,s)$-RdB graph.
        \item $N(k,s)$: maximal length of $(k,s)$-RdB sequences $(=\ell(k,s)+k-1)$.
    \end{itemize}
    \begin{theorem}
        Given $k,s$. Then:
        \begin{align}
            \ell(k,s) = \card{W(k,s)} - \left(\sum_{i=0}^{C}(s-i)\card{W(k-s-i-3,s)}-s\right)
        \end{align}
    \end{theorem}
    where :
    \begin{itemize}
        \item $C=\min(s-1,k-s-2)$.
        \item $W(n,s)$: set of length $n$ sequences containing at most $s$ consecutive bit $0$'s. 
    \end{itemize}
\end{frame}

\begin{frame}{Maximal length of $(k,s)$-RdB sequence}
    \begin{lemma}
        \centering
        $\ell(k,s)\leq\card{W(k,s)} - \left(\sum_{i=0}^{C}(s-i)\card{W(k-s-i-3,s)}-s\right) $
    \end{lemma}
    Observe:
    \begin{itemize}
        \item Vertices: balanced, right-unbalanced, left-unbalanced.
        \item Number of left(right)-unbalanced vertices of form $0^{s}1\ldots10^{i}$ is $\card{W(k-i-s-3,s)}$ with $i\leq C=\min(s-1,k-s-2)$.
    \end{itemize}
    \begin{figure}
        \centering
        \includegraphics[width=0.9\textwidth,height=.3\textheight]{Images/Rate/sketchproof.png}
        \caption{Vertices in RdB graphs.}
        \label{fig:sketchproof}
    \end{figure}
\end{frame}

\section{Rate and Maximal Asymptotic Rate}
\begin{frame}{Definition of rate}
    Recall $N(k,s) = \ell(k,s)+k-1$. Given $k,s$:
    \begin{itemize}
        \item Let $\bfx_{k,s}$: a $(k,s)$-RdB sequence. The (information) rate of $\bfx_{k,s}$:
        \begin{align}
            R_{\bfx_{k,s}} = \dfrac{\log\left(\card{\bfx_{k,s}}\right)}{k}
        \end{align}
        \pause
        \item Maximal rate of $(k,s)$-RdB sequences:
        \begin{align}
            R_{k,s} = \dfrac{\log(N(k,s))}{k}
        \end{align}
        \pause
        \item  Maximal asymptotic rate of $(k,s)$-RdB sequences:
        \begin{align}
            R_{s} = \lim_{n\to\infty}\dfrac{\log(N(k,s))}{k}
        \end{align}
    \end{itemize}
\end{frame}

\begin{frame}{Maximal Asymptotic Rate}
    \begin{theorem}
        \centering $R_{1}= 0.6942$
    \end{theorem}
    which is larger than $0.5$, rate of HdB sequences.
    \begin{theorem}
        \begin{center}
            $R_{s} = \log(\card{\omega})$
        \end{center} 
        
        where $\omega$ is the root of equation: $x^{s+1}-x^{s}-\ldots-x-1=0$ such that $\card{\omega}$ is the largest.
    \end{theorem}
    \textit{Proof}: Approximation: $\card{W(k,s)}\approx a\card{\omega}^{n}$. Hence: 
    $$N(k,s)\approx \card{\omega}^{k-s-2}\left(\sum_{i=0}^{s}a(i+1)\card{w}^{s-i}+\dfrac{s+k-1}{\card{w}^{k-s-2}}\right)$$
    \[\Rightarrow R_{s} = \lim_{n\to\infty}\dfrac{(k-s-2)\log(\card{\omega})}{k} = \log(\card{\omega}).\]
\end{frame}
\section{Encoding and Decoding Algorithm}
\begin{frame}{Encoder}
    \begin{itemize}
        \item Based on lexicographic minimal de Bruijn sequence (granddaddy by Knuth):
        
        Example with order $5$: $0\ 00001\ 00011\ 00101\ 00111\ 01011\ 01111\ 1$
        
        $\rightarrow$ Cut the sequence at the "right place".
        \item Complexity = Complexity to generate granddaddy sequence.
        
        \hspace{1.8cm} = Constant amortized time per symbol~\sfcite{ruskey1992generating}.
        \item The correctness can be proved easily.
    \end{itemize}
    \pause
    Cut at $\bfu = 0^{s+1}1^{k-s-1}$.
    
    \pause
    Example $k=5,s=2:$

    \only<4>{$$0\ 00001\ 00011\ 00101\ 00111\ 01011\ 01111\ 1$$}
    
    \only<5>{$${\color{gray}0\ 00001\ 0} {\color{red}|}0011\ 00101\ 00111\ 01011\ 01111\ 1$$}
    
    \only<6->{$${\color{gray}0\ 00001\ 0} {\color{red}|}0011\ 00101\ 00111\ 01011\ 01111\ 1\ {\color{cyan}00}$$}    
\end{frame}

\begin{frame}{Encoder}
    \begin{algorithm}[H]
    \DontPrintSemicolon
        \SetKwInOut{KwIn}{Input}
        \SetKwInOut{KwOut}{Output}
        \KwIn{$k$, and descending ordered set \L$^{(n)}$. }
        \KwOut{ $(k,s)$-RLL dBs}
        \BlankLine
            % Find the set of Lyndon words  $S=\left\{\lambda_{i}:\ \lambda_{i}\in\mathsf{L}^{(k)}\right\}$
        
        $\mathbf{w}\gets$emptystring\;
        \For{$\lambda \in$\L$^{(n)}$}{
            $\mathbf{w}.prepend(\lambda)$\;
            \If{$\lambda== 0^{s+1}1^{k-s-1}$}{
                $\mathbf{w} = \mathbf{w}[2,\ell]0^{s}$\;
                \textbf{break}\;
            }
        }
        \KwRet{$\mathbf{w}$}
        \caption{Encode (k,s)-RdB}
        \label{alg:encoder}
    \end{algorithm}
    $\text{\L}^{(n)}$: generated by FKM algorithm~\sfcite{fredricksen1978necklaces,fredricksen1986algorithm}.
\end{frame}

\begin{frame}{The Optimality of The Encoder }
    \begin{itemize}
        \item We're proving that our encoder generate the longest $(k,s)$-RdB sequences.
    \end{itemize}
    \begin{example}[$k=5,s=2$]
    $${\color{gray}0\ 00001\ 0} {\color{red}|}\underbrace{0011\ 00101\ 00111\ 01011\ 01111\ 1 {\color{cyan}00}}_{\mathrm{this\ length}=N(k,s)\ ?} $$
    \visible<2->{
    Equivalent to:
        $${\color{gray}0\ 00001\ 0} {\color{red}|}\underbrace{0011\ 00101\ 00111\ 01011\ 01111\ 1}_{\text{How long is this sequence? (=X)}} {\color{cyan}00}$$
        We'll prove that $X=N(k,s)-s$
    }
    \end{example}
    
\end{frame}

\begin{frame}{The Optimality of The Encoder}
$\langle \bfv\rangle$ minimal rotation of $\bfv$ (exp: $\langle110\rangle=011,\ \langle1001\rangle=0011$).

$S(\bfv) = \left\{\bfx\in\Sigma^{\card{\bfv}:\langle\bfx\rangle<\bfv} \right\}$.

\L$^{(n)}$: set of Lyndon words whose length is a divisor of $n$.

\begin{lemma}[Lemma 29\sfcite{kociumaka2016efficient}]
    Let $\bfv$ be a Lyndon word. Define \L$(\bfv) = \{\lambda\in\text{\L}^k:\lambda^{\frac{k}{\card{\lambda}}}\leq\bfv\}$ to be the set of all Lyndon words smaller than $\bfv$ whose length is the divisor of $k$. Then: $\sum_{\lambda\in\text{\L}(\bfv)}\card{\lambda}=\card{S(\bfv)}$.
\end{lemma}

\[\lefteqn{\overbrace{\phantom{0\ 00001\ 0{\color{red}|}0011\ }}^{\mathrm{length}=\card{S(\bfu)}}}
0\ 00001\ \underbrace{0{\color{red}|}0011\ 00101\ 00111\ 01011\ 01111\ 1}_{\text{length needs calculating}}\]
\end{frame}

\begin{frame}{The Optimality of The Encoder}
    \begin{overprint}
        \onslide<1>
        \begin{align}
            \card{S(\bfu)} = 1 + \sum_{t=M}^{k}(k-t+1)\card{C(t-2,s)} + \sum_{t=1}^{k-s}(k-t+1)A_{t}
        \end{align}
        where $\card{C(k,s))} = 2^{k}-\card{W(k,s)}$, $A_{t}=2^{t-2}$, $M=\max(s+2,k-s+1)$
        \onslide<2->
        \centering For $s=1$, $\card{S(\bfu)}=2^{k}-\left(\card{W(k,1)}-\card{W(k-4,1}\right)$
    \end{overprint}
    \only<3-> {
        $k=5,s=1$
    }
    
    \only<3>{
        \[\overbrace{0\ 00001\ 00011\  00101\ 0{\color{red}|}0111\ }^{ 2^{k}-\left(\card{W(k,1)}-\card{W(k-4,1}\right) } 01011\ 01111\ 1\]
    }
    
    \only<4>{
        \[\lefteqn{\overbrace{\phantom{0\ 00001\ 00011\  00101\ 0{\color{red}|}0111\ }}^{ 2^{k}-\left(\card{W(k,1)}-\card{W(k-4,1}\right) } }
        0\ 00001\ 00011\  00101\ \underbrace{0{\color{red}|}0111\ 01011\ 01111\ 1}_{\card{W(k,1)}-\card{W(k-4,1)}+k}\]
    }
    
    \only<5>{
        \[\lefteqn{\overbrace{\phantom{0\ 00001\ 00011\  00101\ 0{\color{red}|}0111\ }}^{ 2^{k}-\left(\card{W(k,1)}-\card{W(k-4,1}\right) } }
        0\ 00001\ 00011\  00101\ \underbrace{\underbrace{0{\color{red}|}0111\ 01011\ 01111\ 1}_{\card{W(k,1)}-\card{W(k-4,1)}+k}}_{=N(k,1)}\]
    }
\end{frame}

\begin{frame}{Decoder}
    \begin{enumerate}
        \item Based on the decoder $\mathcal{D}_{KRR}$ proposed by \citeauthor{kociumaka2016efficient}\sfcite{kociumaka2016efficient}.
        \item Based on the observation:
        \begin{itemize}
            \item $i=\mathcal{D}_{KRR}(\bfu=0^{s+1}1^{k-s-1})$: position of $\bfu$ in granddaddy sequence $\bfx$ of order $k$.
            \item Location of $\bfv$ ($\card{\bfv}=k$) in encoded sequence = Location of $\bfv$ in $\bfx$ - $i$.
            
            \hspace{6.7cm}= $\mathcal{D}_{KRR}(\bfv)-i$.
        \end{itemize}
        \item Complexity = Complexity of $\mathcal{D}_{KRR}$.
    \end{enumerate}
    \begin{columns}
        \visible<2->{
        \begin{column}{.3\textwidth}
            Exp: $k=5,s=1$
            
            then $\bfu=00111$
            
            say: $\bfv=10111$
        \end{column}
        }
        \visible<3->{
        \begin{column}{.7\textwidth}
            \only<3>{
                $0\ 00001\ 00011\ 00101\ 00111\ 01011\ 01111\ 1\ {\color{cyan}0}$
            }
            \only<4>{
                $0\ 00001\ 00011\ 00101\ \underset{i}{\fcolorbox{red}{white}{00111}}\ 01011\ 01111\ 1\ {\color{cyan}0}$
            }
            
            \only<5->{
                $0\ 00001\ 00011\ 00101\ \underset{i}{\fcolorbox{red}{white}{00111}}\ 0101\underset{j}{\fcolorbox{red}{white}{1\ 0111}}1\ 1\ {\color{cyan}0}$
            }
            
            \only<6>{
            
                $\hspace{3.9cm}0111\ \ 0101\ 1\ 0111\ 1\ 1\ {\color{cyan}0}$
            }
            \only<7->{
                
                $\hspace{3.9cm}0111\ \ 0101\underset{j-i}{\fcolorbox{red}{white}{1\ 0111}}1\ 1\ {\color{cyan}0}$
            }
        \end{column}
        }
    \end{columns}        
\end{frame}

\begin{frame}{Decoder}
    Denote $c_{k,s}$ to be the encoded sequence.
    \begin{algorithm}[H]
    \DontPrintSemicolon
        \SetKwInOut{KwIn}{Input}
        \SetKwInOut{KwOut}{Output}
        \KwIn{A word $\bfv=(v_1,\ldots,v_k)$ of length $k$}
        \KwOut{a is the location of $\bfv$ in $\bf\bfc_{k,s}$}
        \BlankLine
        
        $i \gets \mathcal{D}_{KRR}(\bfu_{k,s})$; \\
         \If {$\bfv = 1^{j}0^{k-j}$,}{\KwRet{$n-i+1-(k-j)$};}
         \Else{\KwRet{$\mathcal{D}_{KRR}(\bfv)-i$};}
        \caption{Decode (k,s)-RdB $\bf\bfc_{k,s}$}
        \label{alg:decode}
    \end{algorithm}  
\end{frame}

\section{Conclusion}
\begin{frame}{Summary}
    \begin{block}{A new combinatorial structure}
        \begin{itemize}
            \item Run length limited de Bruijn sequences (RdB).
            \item Open new questions in research.
            \item Potential applications: Quantum communication, DNA storage, \dots.
        \end{itemize}
    \end{block}
    
    \begin{block}{Length and rate}
        \begin{itemize}
            \item Explicit formula for maximal length and maximal asymptotic rate of RdB sequences.
        \end{itemize}
    \end{block}
    
    \begin{block}{Encoding and decoding algorithm}
        \begin{itemize}
            \item Encode in constant amortized time per symbol.
            \item Decode in polynomial time (sub-linear with respect to the length of the whole sequence).
        \end{itemize}
    \end{block}
\end{frame}

\begin{frame}{Future works}
    \begin{block}{More Constraints}
        \begin{itemize}
            \item Weight constraint.
            \item Locally constraint.
            \item \dots
        \end{itemize}
    \end{block}
    
    \begin{block}{Alphabet}
        \begin{itemize}
            \item Extend the size of alphabet.
            \item Alphabet $\left\{A,T,C,G\right\}$ (DNA).
        \end{itemize}
    \end{block}
    
    \begin{block}{Properties}
        \begin{itemize}
            \item Lexicographic minimal RdB sequences ?
            \item How many sequences of the same size ?
            \item \dots
        \end{itemize}
    \end{block}
\end{frame}

\begin{frame}{Publications}
    My publications during the time doing this thesis:
    \begin{itemize}
        \item Yeow Meng Chee, Duc Tu Dao, \textbf{Tien Long Nguyen}, Duy Hoang Ta, Van Khu Vu. "Run Length Limited de Bruijn Sequences for Quantum Communications", The 2022 IEEE International Symposium on Information Theory.
        \item Tran Ba Trung, Lijun Chang, \textbf{Nguyen Tien Long}, Kai Yao, Huynh Thi Thanh Binh. "Verification-Free Approaches to Efficient Locally Densest Subgraph Discovery", The 39th IEEE International Conference on Data Engineering.
    \end{itemize}
\end{frame}

\begin{frame}[allowframebreaks]{References}
    \printbibliography
\end{frame}

% \begin{frame}{Nội dung}
% \tableofcontents
% \end{frame}

% \section{Lý thuyết}
% \subsection{Định lý thặng dư Trung Hoa}
% \subsection{Thuật toán Sàng Eratosthenes}

% \section{Bài toán RemainderGame}
% \subsection{Phát biểu}
% \subsection{Yêu cầu}
% \section{Thuật toán đề xuất}
% \subsection{Trình bày giải thuật}
% \subsection{Tính đúng đắn của giải thuật}
% \subsection{Phân tích source code}
% \subsection{Demo chương trình/kết quả: chụp ảnh kết quả chạy online, Themis}
% \subsection{Mở rộng bài toán}
% \section*{Tài liệu tham khảo}

% % TODO: Book
% \begin{frame}{Tài liệu tham khảo}
%     \vspace{20pt}

%     \bi
%         \item {\color{hilight}[Tutorial] Chinese Remainder Theorem}\\\url{https://codeforces.com/blog/entry/61290}
%         \item  {\color{hilight}Chinese Remainder Theorem}
%         \\ \url{https://cp-algorithms.com/algebra/chinese-remainder-theorem.html}
%         \vspace{10pt}
%         \item {\color{hilight}Divergence of the sum of the reciprocals of the primes}\\\url{https://en.wikipedia.org/wiki/Divergence_of_the_sum_of_the_reciprocals_of_the_primes}
%     \ei
% \end{frame}


% % TODO: What kind of problems we're dealing with: description/input/output
% \begin{frame}{Định lý thặng dư Trung Hoa}
%     \bi
%         \item Nguồn gốc\\
%         Xuất phát từ bài bài toán Hàn Tín điểm binh
%         \begin{center}
%             \includegraphics[scale=0.5]{Slide/Hantindiembinh.png}
%         \end{center}
%     \ei
% \end{frame}

% % TODO: Example problems/solutions
% \begin{frame}{Định lý thặng dư Trung Hoa}
% \bi
%     \item Nguồn gốc
%     \item Phát biểu
%     \\Định lý Thặng dư Trung Hoa giải quyết vấn đề về sự tồn tại nghiệm của bài toán sau:
%     \\\textit{Dạng 2 số : } Cho 2 cặp số nguyên $(a_{1},n_{1})$ và $(a_{2},n_{2})$.Tìm số nguyên $x$ thoả mãn :
%   \begin{center}
% $\left\{
%     \begin{matrix}
%     x \equiv a_{1}\ (mod\ n_{1}\ ) \\
%     x \equiv a_{2}\ (mod\ n_{2}\ )
%     \end{matrix}
% \right.$
% \end{center}

% \ei 
% \end{frame}

% \begin{frame}{Định lý thặng dư Trung Hoa}
% \bi
%     \item Nguồn gốc
%     \item Phát biểu
%     \\Định lý Thặng dư Trung Hoa giải quyết vấn đề về sự tồn tại nghiệm của bài toán sau:
%     \\\textit{Dạng 2 số : } 
%     \\\textit{Dạng N số : }Cho 2 bộ N số nguyên $a_{1},a_{2},...,a_{n}$ và $(n_{1},n_{2},...,n_{N})$. Tìm số nguyên $x$ thoả mãn :
%   \begin{center}
% $\left\{
%     \begin{matrix}
%     x \equiv a_{1}\ (mod\ n_{1}\ ) \\
%     x \equiv a_{2}\ (mod\ n_{2}\ ) \\
%     ... \\
%     x \equiv a_{N}\ (mod\ n_{N}\ ).
%     \end{matrix}
% \right.$
% \end{center}
% \ei 
% \end{frame}

% \begin{frame}{Định lý thặng dư Trung Hoa}
% \bi
%     \item Dạng 2 số : Phương trình có nghiệm khi và chỉ khi \[|a_{1}-a_{2}|\vdots gcd(n_{1},n_{1})\]
%     \item Chứng minh : 
%     \\Hệ phương trình tương đương với phương trình Diofant
%     \[n_{1}(-k_{1})+n_{2}k_{2}=a_{1}-a_{2}\]
%     \\Áp dụng tính chất sau : Với mọi số nguyên $m,n$ luôn tồn tại 2 số nguyên $a,b$ thoả mãn : $ma+nb=gcd(m,n)$ \\( Chứng minh tính bằng cách quy về $gcd(m,n)=1$ và sử dụng hệ thặng dư đày đủ $n.n'$ với $n'\in {1,2,...,m-1}$)
% \ei 
% \end{frame}

% \begin{frame}{Định lý thặng dư Trung Hoa}
% \bi
%     \item Dạng n số : Quy nạp từ dạng 2 số
%     \item Đăc biệt : 
%     \\ Khi các số $n_{i}$ đôi một nguyên tố cùng nhau thì hệ phương trình luôn có nghiệm
% \ei 
% \end{frame}

% \begin{frame}{Thuật toán Sàng Eratosthenes}
% \bi
%     \item Nguồn gốc
%     \\Ban đầu, nhà toán học Eratosthenes sau khi tìm ra thuật toán, đã lấy lá cọ và ghi tất cả các số từ 2 cho đến 100. Ông đã chọc thủng các hợp số và giữ nguyên các số nguyên tố. Bảng số nguyên tố còn lại trông rất giống một cái sàng. Do đó, nó có tên là sàng Eratosthenes
%     \begin{center}
%         \includegraphics[scale=0.65]{Slide/eratosthenes.png}
%     \end{center}
% \ei
% \end{frame}

% \begin{frame}{Bài toán RemainderGame}
% Pari và Arya chơi một trò chơi Tìm phần dư.
% \\Pari chọn 2 số nguyên dương $x$ và $k$ bất kỳ và cho Arya biết số $k$. Arya cần tìm giá trị của $x\ mod\ k$ biết rằng, cho trước $n$ số nguyên dương $c_{1},\ c_{2},\ ...,\ c_{n}$, và Pari sẽ phải cho Arya biết giá trị của $x\ mod\ c_{i}$ nếu Arya muốn. 
% \\Yêu cầu : Với số $x$ bất kỳ, hỏi Arya có thể tìm được $x\ mod\ k$ không ?
% \bi
% \item \textbf{Input}
% \\Dòng đầu chứa số nguyên $n$ và $k$ ($1\leq n,k \leq 1000000)$
% \\Dòng tiếp theo chứa $n$ số nguyên $c_{1},\ c_{2},\ ...,\ c_{n}$
% $(1\leq c_{i}\leq 1000000\ \forall i \in \overline{1,n})$
% \item \textbf{Output}
% \\In ra "Yes" nếu Arya có chiến thuật thắng, ngược lại, in ra "No" 
% \ei

% \end{frame}

% \begin{frame}{Bài toán RemainderGame}
% \bi
% \item Ví dụ
% \begin{center}
%     \includegraphics[scale=0.5]{Slide/exampletest.png}
% \end{center}
% \item \textbf{Giới hạn}
% \\Thời gian : 1s
% \\Bộ nhớ : 256MB
% \ei
% \end{frame}

% \begin{frame}{Bài toán RemainderGame}
% \bi
% \item pseudo-code
% \begin{center}
%     \includegraphics[scale=0.5]{Slide/pseudo-main.png}
% \end{center}
% \ei
% \end{frame}

% \begin{frame}{Bài toán RemainderGame}
% \bi
% \item Tính đúng đắn của giải thuật
% \\Giải thuật là đúng nếu $lcm(c_{1},c_{2},\ ...,\ c_{n})\vdots\ k$ thì có thể tìm được $x(mod\ k)$, ngược lại thì không thể tìm được.

% \begin{center}
%     $\left\{ 
%     \begin{matrix}
%     x \equiv\ r_{1}\ (mod\ c_{1})\\
%     x \equiv\ r_{2}\ (mod\ c_{2})\\ 
%     ... \\
%     x \equiv\ r_{n}\ (mod\ c_{n})\\
%     \end{matrix}
%     \right.$
% \end{center}
% Chứng minh theo 2 chiều
% \\$\bullet$ Nếu $lcm\vdots\ k$ thì tìm được giá trị $x\ mod\ k$
% \\$X$ bất kỳ thoả mãn (*) thì $X\equiv x\ (mod\ k )$(định lý thặng dư Trung Hoa, có $X-x\vdots\ lcm$, mà $lcm\vdots\ k$).
% \\$\bullet$ Ngược lại, để biết được $x\ mod\ k\ \Rightarrow$  $lcm\vdots\ k$.

% \ei
% \end{frame}

% \begin{frame}{Bài toán RemainderGame}

% \bi
% \item Tính đúng đắn của giải thuật
% \\Giải thuật là đúng nếu $lcm(c_{1},c_{2},\ ...,\ c_{n})\vdots\ k$ thì có thể tìm được $x(mod\ k)$, ngược lại thì không thể tìm được.
% \\$\bullet$ Nếu $lcm\vdots\ k$ thì tìm được giá trị $x\ mod\ k$
% \\$\bullet$ Ngược lại, để biết được $x\ mod\ k\ \Rightarrow \ lcm\vdots\ k$.
% \\Mọi nghiệm của phương trình (*) phải có cùng số dư khi chia cho $k$. $A,\ B$, có dạng :
% \begin{center}
%     $\left\{
%     \begin{matrix}
%     A = lcm*a + l\\
%     B = lcm*b + l.
%     \end{matrix}
%     \right.$
% \end{center}
% với $a,l \in \mathds{Z}$.
% Vì $A\equiv B\ (mod\ k)$ nên $lcm(a-b)\vdots\ k$, có thể chọn $a, b$ sao cho $gcd((a-b),k)=1$, khi đó suy ra $lcm\vdots\ k$.
% \ei

% \end{frame}

% \begin{frame}{Bài toán RemainderGame}

% \lstinputlisting[language=C++]{prime_factorization.cpp}

% \end{frame}

% \begin{frame}{Bài toán RemainderGame}

% \lstinputlisting[language=C++]{sieve_of_eratosthenes.cpp}

% \end{frame}

% \begin{frame}{Bài toán RemainderGame}

% \lstinputlisting[language=C++]{is_divide.cpp}

% \end{frame}

% \begin{frame}{Đánh giá giải thuật}
%     \bi
%     \item Sàng Eratosthenes $O(Nlog(log(N)))$
%     \item Phân tích ra thừa số nguyên tố $O(logN)$
%     \item Kiểm tra tính chia hết $O(N)$
%     \item Toàn bộ giải thuật $O(Nlog(log(N)))$
%     \ei
% \end{frame}

% \begin{frame}{Bài toán RemainderGame}
% \bi
% \item Chạy trên themis
% \begin{center}
%     \includegraphics[scale=0.37]{Themis.png}
% \end{center}
% \item Chạy trên Codeforce
% \ei

% \end{frame}
% \begin{frame}{Bài toán RemainderGame}
% \bi
% \item Chạy trên themis
% \item Chạy trên Codeforce
% \begin{center}
%     \includegraphics[scale=0.30]{Statuscodeforce.png}
% \end{center}
% \ei
% \end{frame}

% \begin{frame}{Mở rộng bài toán}
%     Bài toán ban đầu yêu cầu trả lời câu hỏi có thể xác định được $x\mod k$ hay không, tuy nhiên ta có thể nhìn nhận vấn đề theo 1 cách khác như sau. Thay vì trả lời câu hỏi có tìm được $x(mod\ k)$ hay không, câu hỏi đặt ra là với 1 số $x$ thoả mãn : 
% \begin{center}
%     $\left\{
%     \begin{matrix}
%     x\equiv a_{1}\ (mod\ n_{1})\\
%     x\equiv a_{2}\ (mod\ n_{2})\\
%     ...\\
%     x\equiv a_{N}\ (mod\ n_{N})
%     \end{matrix}
%     \right.$
% \end{center}
% Hãy tìm giá trị nhỏ nhất của $x(mod\ k)$
% \end{frame}

% \begin{frame}{Mở rộng}
%     Sử dụng thuật toán tìm nghiệm cuả hệ thặng dư Trung Hoa để tìm nghiệm của phương trình trên $\rightarrow X(mod\ lcm(n_{1},n_{2},\ ...,n_{N}))\rightarrow $ tìm $y$ sao cho hệ phương trình nghiệm nguyên sau có nghiệm:
% \begin{center}
%     $\left\{
%     \begin{matrix}
%     x\equiv X\ (mod\ lcm(n_{1},n_{2},\ ...,n_{N}))\\
%     x\equiv a\ (mod\ k)
%     \end{matrix}
%     \right.$
% \end{center}
% \ \ \ Giả sử tìm được $X$ ( thuật toán Garner có độ phức tạp là $O(N^{2})$ trong trường hợp các số $n_{i}$ đôi một nguyên tố cùng nhau). 
% \\ \ \ \ Tính $d=gcd(lcm(n_{1},n_{2},\ ...,n_{N}), k)$. Vì $lcm(n_{1},n_{2},\ ...,n_{N})$ có thể rất lớn $\rightarrow $ phân tích $lcm(n_{1},n_{2},\ ...,n_{N})$ và $k$ ra thành tích các thừa số nguyên tố r tính $d$ theo định nghĩa.$\rightarrow O(Nlog(log(N)))$.
% \\ \ \ \ Duyệt $a$ từ $1$ đến $k$, ngay khi tìm được $a$ thoả mãn $|a-X| \vdots d$ thì dừng và đây chính là giá trị cần tìm.$\rightarrow O(N)$ ( vì $k\leq N$ )
% \\\indent Như vậy, việc thay đổi yêu cầu đề bài đã làm cho bài toán khó giải hơn, giải thuật có độ phức tạp từ $O(Nlog(log(N)))$ thành $O(N^{2})$
% \end{frame}
\end{document}