\begin{frame}{Summary}
    \begin{block}{A new combinatorial structure}
        \begin{itemize}
            \item Run length limited de Bruijn sequences (RdB).
            % \item Open new questions in research.
            % \item Potential applications: Quantum communication, DNA storage.
            \item Encode and decode efficiently.
        \end{itemize}
    \end{block}
    
    \begin{block}{Length and rate}
        \begin{itemize}
            \item Explicit formula for maximal length and maximal asymptotic rate of RdB sequences.
        \end{itemize}
    \end{block}
    
    \begin{block}{Encoding and decoding algorithm}
        \begin{itemize}
            \item Encode in constant amortized time per symbol.
            \item Decode in polynomial time (sub-linear with respect to the length of the whole sequence).
        \end{itemize}
    \end{block}
\end{frame}

\begin{frame}{Future works}
    \vfill\begin{block}{More Constraints}
        \begin{itemize}
            \item Weight constraint.
            \item Locally constraint.
            \item Extend the size of alphabet.
        \end{itemize}
    \end{block}
    
    % \begin{block}{Alphabet}
    %     \begin{itemize}
            
    %         \item Alphabet $\left\{A,T,C,G\right\}$ (DNA).
    %     \end{itemize}
    % \end{block}
    
    \vfill\begin{block}{Properties}
        \begin{itemize}
            \item Lexicographic minimal RdB sequences ?
            \item How many sequences of the same size ?
            % \item \dots
        \end{itemize}
    \end{block}
\end{frame}

\begin{frame}{Publications}
    My publications during the time doing this thesis:
    \begin{itemize}
        \item Yeow Meng Chee, Duc Tu Dao, \textbf{Tien Long Nguyen}, Duy Hoang Ta, Van Khu Vu. "Run Length Limited de Bruijn Sequences for Quantum Communications". In proceeding of IEEE International Symposium on Information Theory, 2022.
        \item Tran Ba Trung, Lijun Chang, \textbf{Nguyen Tien Long}, Kai Yao, Huynh Thi Thanh Binh. "Verification-Free Approaches to Efficient Locally Densest Subgraph Discovery". Accepted as paper at The 39th IEEE International Conference on Data Engineering, 2023.
    \end{itemize}
\end{frame}

\begin{frame}[allowframebreaks]{References}
    \printbibliography
\end{frame}